This work concerned two experimental searches for the violation of fundamental discrete symmetries in physical systems originating from electron-positron interactions.

The first study was a direct test of the symmetry under reversal in time in transitions of neutral K mesons, performed with quantum-entangled neutral kaon pairs produced in the $e^+e^-\to\phi\to\Ks\Kl$ process. Data collected by the KLOE experiment operating at the DA$\Phi$NE collider in 2004--2005 were studied to select events of the $\Ks\Kl\to\pi e \nu\;3\pi^0$ and $\Ks\Kl\to \pi^+\pi^-\;\pi e\nu$ processes and compare their rates. For the $\Kl\to 3\pi^0$ decay involving only neutral particles, a dedicated reconstruction technique based on trilateration was devised. Rates of each process identified by two time-ordered neutral kaon decays, determined as a function of a difference between kaon decays, were used to measure the asymptotic level of two \Ts-violation sensitive ratios of double kaon decay rates, yielding the values of $R_2 = 1.020 \pm 0.017_{stat} \pm 0.035_{syst}$ and $R_4 = 0.990 \pm 0.017_{stat} \pm 0.039_{syst}$.
In agreement with expectation based on the size of the the dataset used, these results do not reach the sensitivity needed to probe T violation. However, this measurement proves that the required reconstruction and analysis of the data is feasible and prospects exist for a statistically significant test of the T symmetry with a larger dataset collected by the KLOE-2 experiment is certain systematic effects are eliminated.

The second part of this work comprised a demonstration of the feasibility of using the J-PET detector to search for non-vanishing angular correlations in the decays of ortho-positronium atoms, the lightest purely leptonic systems decaying into photons. The trilateration based reconstruction method prepared for $\Kl\to 3\pi^0$ decay at KLOE was adapted to the ortho-positronium annihilations into three photons. Its performance was validated using Monte Carlo simulations proving it may be applied to determination of spin direction of positrons forming the positronium atoms, thus allowing for control of their polarization in the experiment. Moreover, the feasibility of identification of $3\gamma$ events as well as reconstruction of their origin points was demonstrated using a test measurement performed with the J-PET detector.



%%% Local Variables:
%%% TeX-master: "main"
%%% End: 