\chapter{Summary and perspectives}
\label{chapter:conclusions}

The aim of this Thesis was to  pursue the investigations of fundamental discrete symmetries with two new measurements. Firstly, a direct test of the symmetry under time reversal using transitions between flavour and CP-definite states of quantum-entangled neutral K mesons was performed using data collected by the KLOE detector, with a view to realization of a statistically sensitive test with the KLOE-2 experiment.
Secondly, a fesibility study of planned searches of discrete symmetry violations in purely leptonic systems, i.e.\ in the decays of ortho-positronium atoms into photons, manifested as non-vanishing angular correlations of final state photons' momenta, was carried out using the J-PET experimental setup.

As both of these objectives relied on reconstruction of the location and time of neutral particle decays into photons, a novel reconstruction procedure based on trilateration was devised and applied to $\Kl\to 3\pi^0\to 6\gamma$ decays recorded by the KLOE detector and to \ops/$\to 3\gamma$ in J-PET.

It was demonstrated that the trilaterative reconstruction of $\Kl$ decays into neutral pions provides a resolution of $\Kl$ proper decay time at a constant level of about $1.6\:\tau_{S}$, only two times less precise than in case of using drift chamber vertices for reconstruction of kaon decays into charged particles. The possibility to reconstruct $\Kl\to 3\pi^0$ decays at KLOE allowed for extraction of samples of $\Ks\Kl\to \pi e \nu\; 3\pi^0$ and $\Ks\Kl\to\pi^+\pi^-\;\pi e \nu$ from 1.7~fb$^{-1}$ of the $e^+e^-\to\phi\to\Ks\Kl$ data collected by the KLOE experiment in 2004--2005. The extracted event sets, split into subsamples by charge of leptons in the semileptonic kaon decay, were used to construct two ratios of double kaon decay rates expressed as functions of the kaons' proper decay time difference $\Delta t$. A quantitative measure of the level of \Ts~noninvariance was determined as the constant level of these ratios in the $\Delta t \gg \tau_{S}$ with respect to unity. The asymptotic level of the \Ts-violation sensitive ratios was estimated by means of a dedicated maximum likelihood fit. The obtained values:
\begin{eqnarray}
  \label{eq:result_in_conclusions}
  R_2 &= 1.020 \pm 0.017_{stat} \pm 0.035_{syst},\\
  R_4 &= 0.990 \pm 0.017_{stat} \pm 0.039_{syst},
\end{eqnarray}
are in agreement with unity within the achieved precision, insufficient for sensitivity to \Ts-violating effects.
% as expected based on the statistics of KLOE dataset used.
However, the resulting uncertainties and performance of the devised analysis show perspectives for reaching the required sensitivity level with the data of the KLOE-2 experiment (which aims at collecting at least 5~fb$^{-1}$ of $\phi$ decay data) provided that the presently dominant systematic errors are mitigated. The results obtained with KLOE data are still affected by certain background contamination of the extracted data samples and imperfections of the used efficiency estimation method. These problems have been identified and possible solutions to be considered in the KLOE-2 data analysis have been proposed.

In addition to the planned test of the symmetry under time reversal at KLOE-2, the prepared data analysis steps may be used in a direct test of the \CPTs~symmetry with neutral kaons at KLOE and KLOE-2 using a similar principle~\cite{theory:bernabeu-cpt}. Such a test has never been performed to date and work is in progress with a view to establishing the first result of \CPTs~test in transitions of neutral mesons with the KLOE data~\cite{KLOE-2:2017lyj,Gajos:2017tlb}.

In the second study comprised in this Thesis, the feasibility of the searches for non-zero average angular correlations in ortho-positronium decays into three photons was demonstrated using a test measurement featuring a cylindrical aluminum chamber for production of direct $e^+e^-\to 3\gamma$ annihilations. A simple selection of 3$\gamma$ event candidates was proposed and about 1100 candidates were identified in the test data. A dedicated reconstruction method was prepared for such decays, validated using Monte Carlo simulations and applied to the candidate events. Additionally, tomographic images of the annihilation chamber using 2$\gamma$ annihilations were obtained as a~benchmark. Although the limited statistics of the test measurement only allowed for qualitative conclusions on the performance of the 3$\gamma$ event identification and reconstruction, distribution of the obtained annihilation points around the location of the annihilation medium shows good prospects for future measurements of \ops/$\to 3\gamma$ events whose yield is expected at a level higher than in the test measurement by two orders of magnitude. Additionally, as the sensitivity of the trilateration-based reconstruction to timing resolution of the detector was demonstrated, a large improvement of the reconstruction performance should come from enhancements of the J-PET detector calibration and data reconstruction procedures, all of which are being elaborated on at the time of writing of this Thesis.

%%% Local Variables:
%%% TeX-master: "../main"
%%% End: 
