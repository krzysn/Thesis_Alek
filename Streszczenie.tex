\hspace{4em} Niniejsza praca dotyczyła dwóch eksperymentów poszukujących łamania podstawowych symetrii dyskretnych w układach powstających w oddziaływaniach elektron-pozyton.

Pierwszym z rozważanych eksperymentów był bezpośredni test symetrii względem odwrócenia w czasie w przejściach w układzie neutralnych mezonów K, przy pomocy kwantowo splątanych par neutralnych kaonów wytworzonych w procesie $e^+e^-\to\phi\to\Ks\Kl$. Dane zebrane przez eksperyment KLOE prowadzący pomiary na zderzaczu DA$\Phi$NE w latach 2004--2005 zostały przeanalizowane w celu identyfikacji zdarzeń procesów $\Ks\Kl\to\pi e \nu\;3\pi^0$ i $\Ks\Kl\to \pi^+\pi^-\;\pi e\nu$ oraz w celu porównania ich krotności. Na potrzeby identyfikacji zawierającego wyłącznie neutralne cząstki rozpadu $\Kl\to 3\pi^0$ została opracowana dedykowana technika rekonstrukcji oparta na trilateracji. Krotności obserwacji każdego z badanych procesów, wyrażone w funkcji różnicy czasów własnych rozpadów obydwu kaonów, zostały wykorzystane do wyznaczenia asymptotycznego poziomu dwóch stosunków krotności podwójnych rozpadów kaonów neutralnych, wrażliwych na efekty łamania symetrii T. Otrzymane wartości $R_2 = 1.020 \pm 0.017_{stat} \pm 0.035_{syst}$ oraz $R_4 = 0.990 \pm 0.017_{stat} \pm 0.039_{syst}$, zgodnie z przewidywaniami na podstawie rozmiaru użytej próbki danych, nie osiągają dokładności wymaganej do pomiaru stopnia łamania symetrii T. Przeprowadzony pomiar dowodzi jednak, że niezbędna rekonstrukcja oraz analiza danych jest wykonalna i istnieje możliwość przeprowadzenia znaczącego statystycznie testu symetrii T przy użyciu większego zbioru danych zebranego przez eksperyment KLOE-2, pod warunkiem usunięcia niektórych źródeł niepewności systematycznej.

Druga część niniejszej pracy polegała na wykazaniu możliwości użycia detektora J-PET do poszukiwania niezerowych korelacji kątowych w rozpadach atomów orto-pozytonium, najlżejszych całkowicie leptonowych układów ulegających rozpadowi na fotony. Oparta na trilateracji metoda rekonstrukcji przygotowana dla rozpadów $\Kl\to 3\pi^0$ została zaadaptowana do przypadku anihilacji atomów orto-pozytonium na trzy fotony. Weryfikacja jej skuteczności przy pomocy symulacji Monte Carlo wykazała, że może ona zostać wykorzystana do wyznaczenia kierunku spinu pozytonów tworzących orto-pozytonium, tym samym pozwalając na kontrolę ich polaryzacji w eksperymencie. Możliwość identyfikacji anihilacji na trzy fotony oraz rekonstrukcji punktów anihilacji została ponadto wykazana przy pomocy próbnego pomiaru przeprowadzonego przy pomocy detektora J-PET.

%%% Local Variables:
%%% TeX-master: "main"
%%% End:        